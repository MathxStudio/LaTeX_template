\documentclass[UTF8, 12pt]{extarticle}
% The package is best combined in the extarticle environment.
% Certain indentation and spacing issues may arise in the
% article / book / report environments.
\usepackage{allan}
% By default Chinese is turned off. To turn it on, use the option
% \usepackage[chinese]{allan}

\usepackage{lipsum}
\title{\textbf{\color{allandarkblue}Example usage of \url{allan.sty}}}

\begin{document}
    \maketitle
    \tableofcontents
    \begin{mdframed}[style=allancadetbluebox, frametitle={Warning}]
        This package is still in development. Please report any bugs to the author via issues on the \href{https://github.com/MathxStudio/LaTeX_template}{GitHub repository}.
    \end{mdframed}

    \newpage
    \section*{Abstract}
    This document provides an example of how to use the `allan.sty` package. It includes various environments such as theorems, proofs, claims, historical notes, and more. Additionally, it demonstrates the use of custom colors defined in the package.
.
    \listoftheorems[ignoreall,show={theorem,lemma,claim}]

    \part{Introduction}
    \section{Section}
    \subsection{Subsection 1}
    \lipsum[1]
    \begin{theorem}[Chevalley ($\Leftarrow$), Shephard-Todd ($\Rightarrow$)]\label{big theorem on polynomial invariants}
        Suppose $V$ is a vct. space over $\RR$, then $S^G$ is a polynomial ring iff. $\rho(G)\le GL(V)$ is a finite reflection group. Also on $\Leftarrow$ direction, the polynomial ring is generated by exactly $n = \dim V$ indeterminates.
    \end{theorem}
    \begin{proof}[Proof of theorem]
        We first introduce the notion of \defword{Raynolds' operator} $R$ on $V$ as the following:
        Suppose $\mathrm{char}(k)\nmid |G|$, for $G\circlearrowright S$, define $S\to R$, $$f\mapsto f^\sharp \ddef \frac{1}{|G|} \sum_{g\in G} gf\in R$$
            \begin{enumerate}
                \item $\cdot^\sharp$ is linear.
                \item If $g\in R$, then $g^\sharp = g$.
                \item If $p\in S$, $q\in R$, then $(pq)^\sharp = p^\sharp q$.
                \item $\deg p = \deg p^\sharp$, $\forall p\in S$.
            \end{enumerate}
        By \textbf{\color{allanblue}Hilbert Basis thm}, $S$ is Noetherian. Let $$R^+ \ddef \bigoplus_{d > 0} R\cap S_d = \{f\in R \mid \text{constant term of $f = 0$}\} = \ker(R\to k, f\mapsto \text{constant term})$$, and $I\ddef SR^+ = (R^+) \le S$, then $I$ is finitely generated. Pick generators $f_1, \dots, f_r\in R^+$ of $I$ (achievable, c.f. proof of Hilbert Basis thm).

        \begin{claim}\label{claim:generators of R}
                $R$ is generated $f_1, \dots, f_r$ and $1$ \textbf{as an algebra}.
        \end{claim}
        \begin{proof}[Proof of claim]
            WLOG $f\in R$ homogeneous (otherwise decompose into homogeneous parts). Induction on $\deg(f)$.
            \begin{itemize}
                \item If $\deg(f) = 0$, trivial.
                \item If $\deg(f) >0$, then $f\in R^+\subseteq I$. Therefore $\exists s_1, \dots, s_r\in S$ s.t. $f = s_1 f_1 + \dots + s_r f_r$. Hence
                    $$f = f^\sharp = s_1^\sharp f_1 + \dots + s_r^\sharp f_r$$
                    After decomposition, we may assume $s_i^\sharp$ is homogeneous. Then $\deg(s_i^\sharp) = \deg(f) - \deg(f_i) < \deg(f)$. By induction hypothesis, $s_i^\sharp$ is a polynomial in $f_1, \dots, f_r$. Hence $f$ is a polynomial in $f_1, \dots, f_r$.
            \end{itemize}
        \end{proof}
        \begin{hisnote}
            Historically, studying the invariants of a group action is a fundamental topic, which in turn motivated Hilbert to research into polynomial rings and their finite generation properties. This idea was further developed by Noether who introduced the concept of Noetherian rings.
        \end{hisnote}
        To proceed, we also require some facts from field theory:
        \begin{theorem}
            Any two transcendence bases of $K$ have the same cardinality, which is the transcendence degree of $K/k$.
        \end{theorem}
        Let $f_1, \dots, f_r$ be a \textbf{minimal} set of generators of $I$. From \cref{claim:generators of R}, it suffices to show $f_1, \dots, f_r$ are algebraically independent. Once this is done, $R\cong k[y_1, \dots, y_r]\Rightarrow \mathrm{Frac}(R)\cong k(y_1, \dots, y_r)$. So $r = n =$ transcendence degree of $\mathrm{Frac}(R)$.
        \begin{lemma}
            Let $W\le GL(V)$ be a finite reflection group. $S, R$ defined as before. $f_1, \dots, f_r\in R$ s.t. $f_1\notin \sum_{i=2}^{r} Rf_i$. Suppose $g_1, \dots, g_r\in S$ are homogeneous s.t. $g_1 f_1 + \dots + g_r f_r = 0$, then $g_1\in I = SR^+$.
        \end{lemma}
        \begin{proof}[Proof of lemma]
            First, $f_1\notin \sum_{i=2}^{r} Sf_i$, since if $f_1 = \sum_{i=2}^{r} s_i f_i$, then $f_1 = f_1^\sharp = \sum_{i=2}^{r} s_i^\sharp f_i\in \sum_{i=2}^{r} Rf_i$, codntradiction.
        \end{proof}
        Now the proof becomes easier if one notices the following fact:
        \begin{fact}
            The fact is left as an exercise to the reader.
        \end{fact}
        \noindent Therefore the proof is complete.

    \end{proof}

    \subsection{Subsection 2: Colors}
    \begin{enumerate}
        \item {\color{allanred}allanred}
        \item {\color{allangreen}allangreen}
        \item {\color{allanblue}allanblue}
        \item {\color{allandarkblue}allandarkblue}
        \item {\color{allanorange}allanorange}
        \item {\color{allanpurple}allanpurple}
        \item {\color{allancyan}allancyan}
        \item {\color{allanyellow}allanyellow}
    \end{enumerate}

    \subsection{Other environments}
    \begin{convention}[Convention]\label{convention}
        This is a convention.
    \end{convention}
    \begin{convention}\label{another convention}
        This is another convention.
    \end{convention}
    \begin{hypothesis}[Hypothesis]\label{hypothesis}
        This is a hypothesis.
    \end{hypothesis}
    \begin{equation}\label{eqn:big equation}
        E = mc^2
    \end{equation}
    \Cref{convention,eqn:big equation,another convention} smart referencing.
    \begin{lemma}\label{lemma}
        This is a lemma.
    \end{lemma}
    \Reflemma{lemma} and \reflemma{lemma}.
    \begin{example}[Example]\label{example}
        This is an example.
    \end{example}
\end{document}